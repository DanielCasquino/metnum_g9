\section{Conclusiones}

En este estudio se logró simular el escenario de interceptación orbital doble
con un solo impulso, correspondiente al caso 1 que sigue la ruta $S_0
    \rightarrow S_1 \rightarrow S_2$, mediante la generación de un Porkchop plot y
la implementación del método de Gibbs.

El Porkchop nos permitió ver las regiones temporales donde se facilita realiza
la transferencia orbital con bajo error temporal. Esto se puede evidenciar por
la aparición de una franja azul, la cual representa los pares temporales de
$t_0$ y $t_1$ que mejor se ajustan a una buena trayectoria. Este resultado
confirma que existen combinaciones viables de eventos temporales para realizar
una doble interceptación bajo el modelo de dos cuerpos, validando uno de los
principales objetivos del trabajo.

Además, se utilizó el método de Gibbs, el cual estima el vector de velocidad en
el punto intermedio de intersección $S_1$, a partir de las posiciones dadas con
los elementos orbitales. Esta estimación resulta coherente con los valores
esperados para orbitales de este tipo.

Los resultados obtenidos de Porkchop y el método de Gibbs ayudan a una
metodología efectiva para la planificación de trayectoria con múltiples
objetivos.