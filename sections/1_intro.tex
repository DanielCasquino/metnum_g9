\section{Introducción}

La exploración espacial ha avanzado significativamente gracias al desarrollo de tecnologías
que permiten ejecutar misiones más eficientes y de bajo costo. Sin embargo, la planificación de las trayectorias
y cambios de dirección siguen requiriendo una cantidad considerable de apoyo terrestre. En este contexto, surge el interés
por minimizar la cantidad de impulsos para llegar o interceptar a objetivos, una tarea importante en el mantenimiento de satélites
y exploración interplanetaria \parencite{ZHU20162177}.

El trabajo de Xia et al. (2021) propone un método numérico para resolver el problema de intercepción de dos objetivos
con un solo impulso, permitiendo posiciones de impulso e intercepción libres, tanto en órbitas elípticas como hiperbólicas \parencite{xia2021}.
Este enfoque reduce el problema original —de seis variables— a un sistema no lineal de solo dos variables independientes,
resuelto con el método de Newton-Raphson y estimaciones iniciales obtenidas mediante \textit{porkchop plots}.
Además, se extiende a modelos perturbados por el coeficiente J2 utilizando homotopía y corrección diferencial.

Estudios complementarios, como el de Duan y Liu (2020), exploran alternativas al clásico método de porkchop plots,
proponiendo un enfoque bidimensional más eficiente para determinar ventanas de lanzamiento, lo cual puede ser útil
en la etapa de estimación inicial de trayectorias \parencite{DUAN2020965}.

Por otro lado, el uso de propulsión eléctrica en satélites pequeños, como los CubeSats, ha abierto nuevas posibilidades para
misiones autónomas más complejas. Sistemas como los \textit{electrospray thrusters} o los \textit{gridded ion thrusters} ofrecen maniobras
precisas con menos consumo de combustible, lo cual refuerza la relevancia de modelos como el propuesto por Xia et al.\@ para planificar
trayectorias de forma óptima \parencite{oreilly2021}.

De forma complementaria, Zhu y Yan (2016) presentan un esquema de maniobra con dos impulsos tangenciales
en órbitas elípticas, que enfatiza la eficiencia de combustible bajo restricciones de tiempo y geometría, lo que sustenta
la importancia de estudiar intercepciones óptimas desde un punto de vista numérico \parencite{ZHU20162177}.