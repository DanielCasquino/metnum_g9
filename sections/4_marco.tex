\section{Marco Teórico}
\subsection{Definiciones}
\begin{itemize}
    \item \textbf{Nave perseguidora}: nave que realiza los impulsos y busca interceptar a los dos objetos.
\end{itemize}

\subsection{Conceptos}
El problema de interceptación de dos objetivos con un solo impulso es un reto
fundamental en la mecánica orbital, particularmente en las misiones espaciales
que requieren la optimización de maniobras para minimizar el uso de combustible
\parencite{Battin1999}.
\subsubsection{Interceptación Orbital}
La interceptación orbital es el proceso mediante el cual una nave espacial
ajusta su trayectoria para alcanzar un objetivo en el espacio.
\subsubsection{Two-Body Model}
El modelo de dos cuerpos es un modelo simplificado en mecánica orbital que
describe el movimiento (elíptico) de dos cuerpos bajo la influencia mutua de la
gravedad, sin tener en cuenta otras fuerzas externas (como las perturbaciones
atmosféricas) \parencite{Cerf2013}.
\subsubsection{Impulso}
En astrodinámica, el impulso es un cambio abrupto en la velocidad de una nave
espacial, que altera su trayectoria orbital.
\subsubsection{Transferencia}
Las transferencias elípticas son trayectorias cerradas en forma de elipse,
mientras que las transferencias hiperbólicas son trayectorias abiertas que
permiten a un objeto escapar de la atracción de un cuerpo central (como la
Tierra) \parencite{ZHU20162177}.
\subsubsection{Perturbación J2}
La perturbación J2 es el efecto causado por la forma no esférica de la Tierra,
que altera ligeramente las órbitas de los objetos cercanos debido a la
asimetría gravitacional de la Tierra (concentrada en el ecuador).
%
\subsection{Métodos Numéricos}
\subsubsection{Gráfico Porkchop}
Este método se utiliza para visualizar las posibles combinaciones de los
tiempos de impulso $t_0$ y tiempos de intercepción $t_1$ que minimizan el
tiempo total de la misión o el uso de combustible \parencite{DUAN2020965}.

El gráfico de Porkchop calcula el coste asociado a cada combinación de $t_0$ y
$t_1$ mediante una fórmula de la forma:

\begin{equation}
    \begin{split}
        P(t_0, t_1) = \left| t_1 - t_0 - \frac{M_3(t_1) - M_3(t_0)}{n_3} \right| \\
        + \left| t_2 - t_0 - \frac{M_3(t_2) - M_3(t_0)}{n_3} \right|
    \end{split}
\end{equation}

\begin{itemize}
    \item $M_3(t)$: anomalía media en el tiempo $t$.
    \item $n_3$: movimiento medio de la órbita de la nave perseguidora.
    \item $t_2$: tiempo de intercepción con el segundo objetivo.
\end{itemize}

\begin{center}
    \includegraphics[width=0.4\textwidth]{porkchop_ejemplo.png}
    \captionof{figure}{Ejemplo de Porkchop plot.} \parencite{eagle2025}
\end{center}

\subsubsection{Método de Gibbs}
Una vez que se han obtenido las estimaciones iniciales de $t_0$ y $t_1$ a
partir del gráfico Porkchop, se utiliza el Método de Gibbs para determinar la
órbita de la nave perseguidora. Gracias a esto, se reduce la cantidad de
variables independientes de seis a dos. Este concepto se explica más a
profundidad en la sección de formulación del problema.

El método de Gibbs calcula la velocidad orbital de la nave a partir de tres
vectores de posición conocidos de la nave y los dos objetivos
\parencite{HE2010265}. A partir de los tres vectores de posición $r_1$, $r_2$ y
$r_3$, se calcula la velocidad de la nave perseguidora en la órbita de
intercepción con la siguiente fórmula:

\begin{equation}
    v_{3,tk} = \sqrt{\frac{\mu}{ND} \left( \frac{D \ast r_{3,tk}}{r_{3,tk}} + S \right)}
\end{equation}

Donde:
\begin{itemize}
    \item $v_{3,tk}$: vector de velocidad de la nave perseguidora en el tiempo $t_k$.
    \item $\mu$: constante gravitacional de la Tierra.
    \item N, D, y S\@: valores derivados del producto cruz entre los vectores de
          posición.
\end{itemize}

\subsubsection{Método de Newton-Raphson}
El Método de Newton-Raphson es un algoritmo iterativo utilizado para resolver
sistemas de ecuaciones no lineales, y es esencial para refinar las soluciones
obtenidas con los métodos anteriores. En este caso, se utiliza para ajustar las
variables $t_0$, $t_1$, $t_2$ y el vector de impulso $v$ hasta que se cumplan
las condiciones de interceptación.

Fórmula de iteración:
\[
    x_{n+1} = x_n - J{(x_n)}^{-1} \ast F(x_n)
\]

Donde:
\begin{itemize}
    \item $x_n$ es el vector de valores actuales de las variables.
    \item $J$ es la matriz Jacobiana de las ecuaciones no lineales.
    \item $F(x_n)$ es el vector de funciones (las ecuaciones de las posiciones y tiempos de interceptación).
\end{itemize}
