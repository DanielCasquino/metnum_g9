\section{Recomendaciones}
\begin{itemize}
    \item \textbf{Incorporar perturbaciones en futuras versiones del modelo}: Para aumentar la
          precisión, se recomienda incluir efectos como el término J2, resistencia
          atmosférica o perturbaciones solares en simulaciones futuras, especialmente en
          misiones prolongadas o de mayor sensibilidad dinámica.
    \item \textbf{Extender el modelo a misiones interplanetarias o multiobjetivo}: El modelo puede adaptarse a misiones más complejas que involucren múltiples intercepciones en diferentes planos orbitales o destinos fuera de la órbita terrestre.
    \item \textbf{Aplicar el modelo en misiones de defensa espacial o inspección satelital}: Dada su capacidad para planificar trayectorias eficientes con mínimo impulso, el modelo es útil en misiones de interceptación de satélites, maniobras evasivas o misiones de servicio y desorbitación.
    \item \textbf{Comparar con otros métodos de resolución no lineal}: Se recomienda explorar métodos alternativos como gradiente conjugado, Levenberg-Marquardt o métodos metaheurísticos (ej. algoritmos genéticos) para resolver el sistema de ecuaciones no lineales, comparando tiempos de convergencia y estabilidad.
    \item \textbf{Desarrollar una interfaz gráfica interactiva en MATLAB}: Implementar una interfaz que permita visualizar Porkchop plots y trayectorias de forma dinámica facilitaría su uso didáctico o su aplicación por parte de operadores de misión.
    \item \textbf{Validar el modelo con escenarios reales de misiones espaciales}: Se sugiere aplicar el modelo a trayectorias reales (por ejemplo, misiones de la NASA o SpaceX) para validar su precisión y utilidad práctica en planificación de maniobras reales.
\end{itemize}
